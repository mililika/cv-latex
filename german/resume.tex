\documentclass[10pt, letterpaper]{article}

% Packages:
\usepackage[
    ignoreheadfoot, % set margins without considering header and footer
    top=1 cm, % seperation between body and page edge from the top
    bottom=1 cm, % seperation between body and page edge from the bottom
    left=2 cm, % seperation between body and page edge from the left
    right=2 cm, % seperation between body and page edge from the right
    footskip=1.0 cm, % seperation between body and footer
    % showframe % for debugging 
]{geometry} % for adjusting page geometry
\usepackage{titlesec} % for customizing section titles
\usepackage{tabularx} % for making tables with fixed width columns
\usepackage{array} % tabularx requires this
\usepackage[dvipsnames]{xcolor} % for coloring text
\definecolor{primaryColor}{RGB}{0, 0, 0} % define primary color
\usepackage{enumitem} % for customizing lists
\usepackage{fontawesome5} % for using icons
\usepackage{amsmath} % for math
\usepackage[
    pdftitle={John Doe's CV},
    pdfauthor={John Doe},
    pdfcreator={LaTeX with RenderCV},
    colorlinks=true,
    urlcolor=primaryColor
]{hyperref} % for links, metadata and bookmarks
\usepackage[pscoord]{eso-pic} % for floating text on the page
\usepackage{calc} % for calculating lengths
\usepackage{bookmark} % for bookmarks
\usepackage{lastpage} % for getting the total number of pages
\usepackage{changepage} % for one column entries (adjustwidth environment)
\usepackage{paracol} % for two and three column entries
\usepackage{ifthen} % for conditional statements
\usepackage{needspace} % for avoiding page brake right after the section title
\usepackage{iftex} % check if engine is pdflatex, xetex or luatex

% Ensure that generate pdf is machine readable/ATS parsable:
\ifPDFTeX
    \input{glyphtounicode}
    \pdfgentounicode=1
    \usepackage[T1]{fontenc}
    \usepackage[utf8]{inputenc}
    \usepackage{lmodern}
\fi

\usepackage{charter}

% Some settings:
\raggedright
\AtBeginEnvironment{adjustwidth}{\partopsep0pt} % remove space before adjustwidth environment
\pagestyle{empty} % no header or footer
\setcounter{secnumdepth}{0} % no section numbering
\setlength{\parindent}{0pt} % no indentation
\setlength{\topskip}{0pt} % no top skip
\setlength{\columnsep}{0.15cm} % set column seperation
\pagenumbering{gobble} % no page numbering

\titleformat{\section}{\needspace{4\baselineskip}\bfseries\large}{}{0pt}{}[\vspace{1pt}\titlerule]

\titlespacing{\section}{
    % left space:
    -1pt
}{
    % top space:
    0.3 cm
}{
    % bottom space:
    0.2 cm
} % section title spacing

\renewcommand\labelitemi{$\vcenter{\hbox{\small$\bullet$}}$} % custom bullet points
\newenvironment{highlights}{
    \begin{itemize}[
        topsep=0.10 cm,
        parsep=0.10 cm,
        partopsep=0pt,
        itemsep=0pt,
        leftmargin=0 cm + 10pt
    ]
}{
    \end{itemize}
} % new environment for highlights


\newenvironment{highlightsforbulletentries}{
    \begin{itemize}[
        topsep=0.10 cm,
        parsep=0.10 cm,
        partopsep=0pt,
        itemsep=0pt,
        leftmargin=10pt
    ]
}{
    \end{itemize}
} % new environment for highlights for bullet entries

\newenvironment{onecolentry}{
    \begin{adjustwidth}{
        0 cm + 0.00001 cm
    }{
        0 cm + 0.00001 cm
    }
}{
    \end{adjustwidth}
} % new environment for one column entries

\newenvironment{twocolentry}[2][]{
    \onecolentry
    \def\secondColumn{#2}
    \setcolumnwidth{\fill, 4.5 cm}
    \begin{paracol}{2}
}{
    \switchcolumn \raggedleft \secondColumn
    \end{paracol}
    \endonecolentry
} % new environment for two column entries

\newenvironment{threecolentry}[3][]{
    \onecolentry
    \def\thirdColumn{#3}
    \setcolumnwidth{, \fill, 4.5 cm}
    \begin{paracol}{3}
    {\raggedright #2} \switchcolumn
}{
    \switchcolumn \raggedleft \thirdColumn
    \end{paracol}
    \endonecolentry
} % new environment for three column entries

\newenvironment{header}{
    \setlength{\topsep}{0pt}\par\kern\topsep\centering\linespread{1.5}
}{
    \par\kern\topsep
} % new environment for the header

\newcommand{\placelastupdatedtext}{% \placetextbox{<horizontal pos>}{<vertical pos>}{<stuff>}
  \AddToShipoutPictureFG*{% Add <stuff> to current page foreground
    \put(
        \LenToUnit{\paperwidth-2 cm-0 cm+0.05cm},
        \LenToUnit{\paperheight-1.0 cm}
    ){\vtop{{\null}\makebox[0pt][c]{
        \small\color{gray}\textit{Last updated in September 2024}\hspace{\widthof{Last updated in September 2024}}
    }}}%
  }%
}%

% save the original href command in a new command:
\let\hrefWithoutArrow\href

% new command for external links:


\begin{document}
    \newcommand{\AND}{\unskip
        \cleaders\copy\ANDbox\hskip\wd\ANDbox
        \ignorespaces
    }
    \newsavebox\ANDbox
    \sbox\ANDbox{$|$}

    \begin{header}
        \fontsize{25 pt}{25 pt}\selectfont Mykhailo Kazymyr

        \vspace{5 pt}

        \normalsize
        \mbox{\hrefWithoutArrow{mailto:mykhailokazymyr@gmail.com}{mykhailokazymyr@gmail.com}}%
        \kern 5.0 pt%
        \AND%
        \kern 5.0 pt%
        \mbox{\hrefWithoutArrow{tel:+49 172 5722847}{+49-172-5722847}}%
        \kern 5.0 pt%
        \AND%
        \kern 5.0 pt%
        \mbox{\hrefWithoutArrow{https://www.linkedin.com/in/mykhailo-kazymyr-a6524624a/}{\textit{linkedin}}}%
        \kern 5.0 pt%
        \AND%
        \kern 5.0 pt%
        \mbox{\hrefWithoutArrow{https://github.com/mililika}{\textit{github}}}%
    \end{header}

    \vspace{5 pt - 0.3 cm}

    \section{Über mich}
    Ich bin Softwareentwickler und Wirtschaftsinformatik-Student an der TUM! Ich verfüge über gute praktische Erfahrung mit Java/Spring Boot im Backend und React/TypeScript im Frontend. Ich bin ein großer Fan von Computern und Elektronik und liebe es, tief in die Funktionsweise der Dinge in unserer Welt einzutauchen.

    \section{Berufserfahrung}
    \begin{twocolentry}{
        April 2023 – aktuell
    }
        \textbf{Full Stack Entwickler}, retcor GmbH -- Remote, Deutschland, Teilzeit\end{twocolentry}

    \vspace{0.20 cm}

    \begin{onecolentry}
        \begin{highlights}
            \item Entwicklung einer Full-Stack-Anwendung für Berichtsberechnungen, Implementierung von Client- und Server-Funktionalitäten
            \item Erstellung eines umfassenden CSR-Clients mit React, einschließlich detaillierter Ansichten für Tabellenmanipulation und Excel-Tabellenanalyse
            \item Implementierung einer sicheren API-Kommunikationsschicht mit Authentifizierung und Autorisierung basierend auf Benutzerlizenzen
        \end{highlights}
    \end{onecolentry}

    \vspace{0.2 cm}
    Technologien: Java, Spring Boot, PostgreSQL, JUnit, Keycloak, React, TypeScript, Jest, Material-UI, Redux \\
    \vspace{0.1 cm}
    Hinweis: Projekt befindet sich derzeit in der Entwicklungsphase

    \vspace{0.3 cm}
    
    \begin{twocolentry}{
        Juni 2024 – Okt 2024
    }
        \textbf{Full Stack Entwickler}, \hrefWithoutArrow{https://pitchpower.ai/}{\textit{PitchPower.ai}} -- Remote, USA, Teilzeit\end{twocolentry}

    \vspace{0.20 cm}

    \begin{onecolentry}
        \begin{highlights}
            \item UI-Redesign-Initiative zur Verbesserung der Angebotserstellungsplattform für B2B- und P2B-Sektoren
            \item Optimierung der Cloud Functions Performance und Implementierung von Firestore-Autorisierungsregeln
            \item Entwicklung und Integration von Team-Funktionen unter Beibehaltung der Systemskalierbarkeit
            \item Verbesserung der Plattformsicherheit und des Datenmanagements durch Firebase-Implementierung
        \end{highlights}
    \end{onecolentry}

    \vspace{0.20 cm}
    Technologien: React, Next.js, Vercel, Tailwind CSS, Firebase, Stripe, OpenAI

    \vspace{0.3 cm}
    
    \begin{twocolentry}{
        Mai 2023 – Aug 2023
    }
        \textbf{Frontend Entwickler}, \hrefWithoutArrow{https://writemore.io/}{\textit{WriteMore.io}} -- Remote, USA, Teilzeit\end{twocolentry}

    \vspace{0.20 cm}

    \begin{onecolentry}
        \begin{highlights}
            \item Durchführung eines umfassenden UI-Redesigns zur Verbesserung der Benutzererfahrung und Plattformästhetik
            \item Implementierung eines Benutzerbenachrichtigungssystems und Statistik-Tracking-Funktionalität
            \item Fokus auf Performance und Reaktionsfähigkeit über mehrere Geräte hinweg
        \end{highlights}
    \end{onecolentry}

    \vspace{0.20 cm}
    Technologien: React, TypeScript, Firebase, Tailwind CSS, Next.js

    \section{Ausbildung}

        \begin{twocolentry}{
            Okt 2023 – Sept 2026
        }
            \textbf{Technische Universität München}, B.Sc. Wirtschaftsinformatik\end{twocolentry}

        \vspace{0.10 cm}
        \begin{onecolentry}
            \begin{highlights}
                \item Absolviere ein umfassendes Studienprogramm, das Softwareentwicklung mit betriebswirtschaftlichen Grundlagen verbindet, einschließlich Controlling, Statistik und Business Analysis
                \item Technische Kernkurse: Praktische Programmierung (Java), Einführung in Software Engineering (Python, Java, Spring Boot), Software Engineering für Geschäftsanwendungen (Spring Boot, Spring Security), Datenstrukturen und Algorithmen, Datenbanksysteme, IT-Netzwerke, IT-Sicherheit
            \end{highlights}
        \end{onecolentry}

        \vspace{0.20 cm}

        \begin{twocolentry}{
            Okt 2021 – Sept 2023
        }
            \textbf{Technische Universität München}, B.Sc. Informatik\end{twocolentry}
        
        \vspace{0.20 cm}

        \begin{twocolentry}{
            Sept 2020 – Juni 2021
        }
            \textbf{Technische Universität der Ukraine "KPI"}, B.Sc. Informatik (1 Jahr)\end{twocolentry}
    
    \section{Nebenprojekte}
        
    \begin{twocolentry}{
        Apr 2024 – Juni 2024
    }
        \textbf{Loquela - Sprachlern-Assistent}, \hrefWithoutArrow{https://loquela.mililika.uk}{\textit{loquela.mililika.uk}} \end{twocolentry}
    \vspace{0.20 cm}
    \begin{onecolentry}
        \begin{highlights}
            \item Entwicklung einer adaptiven Sprachlernplattform, die personalisierte Übungen für Lesen, Schreiben und Verständnis basierend auf dem Niveau des Benutzers erstellt
            \item Implementierung eines umfassenden Benutzerstatistik- und Fortschrittsverfolgungssystems zur Verbesserung der Lernerfahrung
            \item Erstellung eines KI-gestützten Aufgabengenerierungssystems für maßgeschneiderte Lernmaterialien
            \item Entwicklung eines responsiven Designs für nahtlose Erfahrung auf Smartphone, Tablet und Laptop
        \end{highlights}
    \end{onecolentry}
    \vspace{0.2 cm}
    Technologien: React, TypeScript, Firebase (Authentication, Realtime Database, Cloud Functions), OpenAI, i18n Library für mehrsprachige Unterstützung (Ukrainisch, Englisch, Deutsch)

    \vspace{0.4 cm}

    \begin{twocolentry}{}
        \textbf{Persönlicher Homeserver} \end{twocolentry}
    \vspace{0.20 cm}
    \begin{onecolentry}
        \begin{highlights}
            \item Konfiguration und Wartung eines Mini-PCs mit Ubuntu Server für persönliche Cloud-Dienste und Webhosting
            \item Implementierung eines selbstgehosteten Fotodienstes und Dateiverwaltungssystems für persönliche Datensouveränität
            \item Einrichtung sicheres Website-Hosting mittels Cloudflare Tunnels mit Docker-Containerisierung
            \item Bereitstellung und Verwaltung mehrerer Gameserver, dabei praktische Erfahrung in der Serveradministration gesammelt
            \item Aufbau einer skalierbaren Infrastruktur mit Docker Compose für einfaches Service-Management und Deployment
        \end{highlights}
    \end{onecolentry}
    \vspace{0.2 cm}
    Technologien: Docker, Docker Compose, Linux, Cloudflare, Tunnels
    
    \section{Technologien}
        \begin{onecolentry}
            \textbf{Programmiersprachen:} Java, JavaScript, TypeScript, SQL
        \end{onecolentry}

        \vspace{0.2 cm}

        \begin{onecolentry}
            \textbf{Technologien:} Spring, Spring Boot, Spring Security, React, Firebase, Next.js, Redux, JUnit, MySQL, PostgreSQL, Git, Github, Keycloak, Vim, Docker, Docker Compose, Linux
        \end{onecolentry}

    \section{Zusätzliches}

    \begin{onecolentry}
        Ich habe diesen Lebenslauf in \LaTeX{} geschrieben! Ich liebe es, neue technische Tools zu lernen, die das Programmieren interessanter und effizienter machen. Die Entdeckung von Vim-Befehlen hat mir eine völlig neue Welt eröffnet - selbst nach zwei Jahren lerne ich immer noch neue Tricks und werde effizienter damit. Diese Art der kontinuierlichen Verbesserung bereitet mir beim Programmieren große Freude und motiviert mich, mich beruflich weiterzuentwickeln.
    \end{onecolentry}

    \vspace{0.1 cm}
    \begin{onecolentry}
        Den Code für diesen Lebenslauf finden Sie auf meiner Github-Seite \texttt{:)}
    \end{onecolentry}

    \vspace{0.1 cm}
    \begin{onecolentry}
    Ich bin die meiste Zeit erreichbar und freue mich von Ihnen zu hören! Sie können mich gerne auf Englisch (Fließend), Deutsch (Fließend) oder Ukrainisch (Muttersprache) kontaktieren.
    \end{onecolentry}

\end{document}%

