\documentclass[10pt, letterpaper]{article}

% Packages:
\usepackage[
    ignoreheadfoot,
    top=1 cm,
    bottom=1 cm,
    left=2 cm,
    right=2 cm,
    footskip=1.0 cm,
]{geometry}
\usepackage[utf8]{inputenc}
\usepackage[T2A]{fontenc}
\usepackage[english, ukrainian]{babel}
\usepackage{titlesec}
\usepackage{tabularx}
\usepackage{array}
\usepackage[dvipsnames]{xcolor}
\definecolor{primaryColor}{RGB}{0, 0, 0}
\usepackage{enumitem}
\usepackage{fontawesome5}
\usepackage{amsmath}
\usepackage[
    pdftitle={Mykhailo Kazymyr's CV},
    pdfauthor={Mykhailo Kazymyr},
    pdfcreator={LaTeX with RenderCV},
    colorlinks=true,
    urlcolor=primaryColor
]{hyperref}
\usepackage[pscoord]{eso-pic}
\usepackage{calc}
\usepackage{bookmark}
\usepackage{lastpage}
\usepackage{changepage}
\usepackage{paracol}
\usepackage{ifthen}
\usepackage{needspace}
\usepackage{iftex}

% Some settings:
\raggedright
\AtBeginEnvironment{adjustwidth}{\partopsep0pt} % remove space before adjustwidth environment
\pagestyle{empty} % no header or footer
\setcounter{secnumdepth}{0} % no section numbering
\setlength{\parindent}{0pt} % no indentation
\setlength{\topskip}{0pt} % no top skip
\setlength{\columnsep}{0.15cm} % set column seperation
\pagenumbering{gobble} % no page numbering

\titleformat{\section}{\needspace{4\baselineskip}\bfseries\large}{}{0pt}{}[\vspace{1pt}\titlerule]

\titlespacing{\section}{
    % left space:
    -1pt
}{
    % top space:
    0.3 cm
}{
    % bottom space:
    0.2 cm
} % section title spacing

\renewcommand\labelitemi{$\vcenter{\hbox{\small$\bullet$}}$} % custom bullet points
\newenvironment{highlights}{
    \begin{itemize}[
        topsep=0.10 cm,
        parsep=0.10 cm,
        partopsep=0pt,
        itemsep=0pt,
        leftmargin=0 cm + 10pt
    ]
}{
    \end{itemize}
} % new environment for highlights


\newenvironment{highlightsforbulletentries}{
    \begin{itemize}[
        topsep=0.10 cm,
        parsep=0.10 cm,
        partopsep=0pt,
        itemsep=0pt,
        leftmargin=10pt
    ]
}{
    \end{itemize}
} % new environment for highlights for bullet entries

\newenvironment{onecolentry}{
    \begin{adjustwidth}{
        0 cm + 0.00001 cm
    }{
        0 cm + 0.00001 cm
    }
}{
    \end{adjustwidth}
} % new environment for one column entries

\newenvironment{twocolentry}[2][]{
    \onecolentry
    \def\secondColumn{#2}
    \setcolumnwidth{\fill, 4.5 cm}
    \begin{paracol}{2}
}{
    \switchcolumn \raggedleft \secondColumn
    \end{paracol}
    \endonecolentry
} % new environment for two column entries

\newenvironment{threecolentry}[3][]{
    \onecolentry
    \def\thirdColumn{#3}
    \setcolumnwidth{, \fill, 4.5 cm}
    \begin{paracol}{3}
    {\raggedright #2} \switchcolumn
}{
    \switchcolumn \raggedleft \thirdColumn
    \end{paracol}
    \endonecolentry
} % new environment for three column entries

\newenvironment{header}{
    \setlength{\topsep}{0pt}\par\kern\topsep\centering\linespread{1.5}
}{
    \par\kern\topsep
} % new environment for the header

\newcommand{\placelastupdatedtext}{% \placetextbox{<horizontal pos>}{<vertical pos>}{<stuff>}
  \AddToShipoutPictureFG*{% Add <stuff> to current page foreground
    \put(
        \LenToUnit{\paperwidth-2 cm-0 cm+0.05cm},
        \LenToUnit{\paperheight-1.0 cm}
    ){\vtop{{\null}\makebox[0pt][c]{
        \small\color{gray}\textit{Last updated in September 2024}\hspace{\widthof{Last updated in September 2024}}
    }}}%
  }%
}%

% save the original href command in a new command:
\let\hrefWithoutArrow\href



\begin{document}
    \newcommand{\AND}{\unskip
        \cleaders\copy\ANDbox\hskip\wd\ANDbox
        \ignorespaces
    }
    \newsavebox\ANDbox
    \sbox\ANDbox{$|$}

    \begin{header}
        \fontsize{25 pt}{25 pt}\selectfont Михайло Казимир

        \vspace{5 pt}

        \normalsize
        \kern 5.0 pt%
        \mbox{\hrefWithoutArrow{mailto:mykhailokazymyr@gmail.com}{mykhailokazymyr@gmail.com}}%
        \kern 5.0 pt%
        \AND%
        \kern 5.0 pt%
        \mbox{\hrefWithoutArrow{tel:+49 172 5722847}{+49-172-5722847}}%
        \kern 5.0 pt%
        \AND%
        \kern 5.0 pt%
        \mbox{\hrefWithoutArrow{https://www.linkedin.com/in/mykhailo-kazymyr-a6524624a/}{\textit{linkedin}}}%
        \kern 5.0 pt%
        \AND%
        \kern 5.0 pt%
        \mbox{\hrefWithoutArrow{https://github.com/mililika}{\textit{github}}}%
    \end{header}

    \vspace{5 pt - 0.3 cm}

    \section{Про мене}
    Я — розробник програмного забезпечення та студент інформаційних систем у TUM. Маю ґрунтовний практичний досвід у розробці бекенду на Java/Spring Boot та фронтенду на React/TypeScript. Захоплююся комп’ютерами та електронікою,    люблю розбиратися в тому, як усе влаштовано, і глибше занурюватися в принципи роботи технологій.
    \section{Досвід роботи}
    \begin{twocolentry}{
        04.2023 – поточно
    }
        \textbf{Full Stack Developer}, retcor GmbH -- Віддалено, Німеччина, Part-Time\end{twocolentry}

    \vspace{0.20 cm}

    \begin{onecolentry}
        \begin{highlights}
            \item Розробка повноцінного застосунку для розрахунку звітів із клієнтською та серверною частинами
            \item Створення CSR-клієнта на React із детальним табличним представленням звітів і можливістю редагування
            \item Реалізація клієнт-серверної комунікації через REST API з авторизацією
            \item Налаштування сесій користувачів, кукі, JWT-токенів та інтеграція сервісу автентифікації Keycloak
        \end{highlights}
    \end{onecolentry}

    \vspace{0.2 cm}
    Технології: Java, Spring Boot, PostgreSQL, JUnit, Keycloak, React, TypeScript, Jest, Material-UI, Redux \\
    \vspace{0.1 cm}
    Примітка: Проект наразі знаходиться в розробці

    \vspace{0.3 cm}
    
    \begin{twocolentry}{
        06.2024 – 10.2024
    }
        \textbf{Full Stack Developer}, \hrefWithoutArrow{https://pitchpower.ai/}{\textit{PitchPower.ai}} -- Віддалено, США, Part-Time\end{twocolentry}

    \vspace{0.20 cm}

    \begin{onecolentry}
        \begin{highlights}
            \item Оновлення та покращення інтерфейсу платформи для генерації комерційних пропозицій на основі штучного інтелекту для B2B та P2B
            \item Оптимізація продуктивності Cloud Functions та налаштування правил авторизації у Firestore
            \item Реалізація та інтеграція командного функціоналу з урахуванням масштабованості системи
        \end{highlights}
    \end{onecolentry}

    \vspace{0.20 cm}
    Технології: React, Next.js, Vercel, Tailwind CSS, Firebase, Stripe, OpenAI

    \vspace{0.3 cm}
    
    \begin{twocolentry}{
        05.2023 – 09.2023
    }
        \textbf{Front End Developer}, \hrefWithoutArrow{https://writemore.io/}{\textit{WriteMore.io}} -- Віддалено, США, Part-Time\end{twocolentry}

    \vspace{0.20 cm}

    \begin{onecolentry}
        \begin{highlights}
            \item Повний редизайн інтерфейсу для покращення UX/UI
            \item Реалізація системи сповіщень і відстеження прогресу користувачів
            \item Оптимізація продуктивності та адаптація під різні пристрої
        \end{highlights}
    \end{onecolentry}

    \vspace{0.20 cm}
    Технології: React, TypeScript, Firebase, Tailwind CSS, Next.js

    \section{Освіта}

        \begin{twocolentry}{
            10.2023 – 09.2026
        }
            \textbf{Технічний університет Мюнхена}, Бакалавр інформаційних систем\end{twocolentry}

        \vspace{0.10 cm}
        \begin{onecolentry}
            \begin{highlights}
                \item Комплексна програма, що поєднує розробку ПЗ та бізнес-аналіз
                \item Основні курси: програмування (Java, Python, Spring Boot), алгоритми та структури даних, бази даних, ІТ-мережі та ІТ-безпека
            \end{highlights}
        \end{onecolentry}

        \vspace{0.20 cm}

        \begin{twocolentry}{
            10.2021 – 09.2023
        }
            \textbf{Технічний університет Мюнхена}, Бакалавр комп'ютерних наук\end{twocolentry}
        
        \vspace{0.20 cm}

        \begin{twocolentry}{
            09.2020 – 06.2021
        }
            \textbf{КПІ ім. Ігоря Сікорського}, Бакалавр інформаційних систем (1 рік)\end{twocolentry}
    
    \section{Особисті проекти}
        
    \begin{twocolentry}{
        04.2024 – 06. 2024
    }
        \textbf{Loquela - Асистент вивчення мов}, \hrefWithoutArrow{https://loquela.mililika.uk}{\textit{loquela.mililika.uk}} \end{twocolentry}
    \vspace{0.20 cm}
    \begin{onecolentry}
        \begin{highlights}
            \item Створив платформу для вивчення мов із персоналізованими вправами
            \item Впровадив систему статистики та відстеження прогресу
            \item Реалізував генерацію завдань за допомогою ШІ
            \item Забезпечив адаптивний дизайн для різних пристроїв
        \end{highlights}
    \end{onecolentry}
    \vspace{0.2 cm}
    Технології: React, TypeScript, Firebase (Authentication, Realtime Database, Cloud Functions), OpenAI, i18n Library для багатомовної підтримки (українська, англійська, німецька)

    \vspace{0.7 cm}

    \begin{twocolentry}{}
        \textbf{Персональний домашній сервер} \end{twocolentry}
    \vspace{0.20 cm}
    \begin{onecolentry}
        \begin{highlights}
            \item Налаштував та підтримую міні-ПК на Ubuntu Server для персональних хмарних сервісів та веб-хостингу
            \item Впровадив самостійно розміщений фотосервіс та систему управління файлами для особистого контролю даних
            \item Налаштував безпечний хостинг веб-сайтів з використанням Cloudflare тунелів та контейнеризації Docker
            \item Розгорнув та керував кількома ігровими серверами, отримавши практичний досвід адміністрування серверів
            \item Створив масштабовану інфраструктуру з використанням Docker Compose для зручного управління та розгортання сервісів
        \end{highlights}
    \end{onecolentry}
    \vspace{0.2 cm}
    Технології: Docker, Docker Compose, Linux, Cloudflare, Tunnels
    
    \section{Технології}
        \begin{onecolentry}
            \textbf{Мови програмування:} Java, JavaScript, Typescript, SQL
        \end{onecolentry}

        \vspace{0.2 cm}

        \begin{onecolentry}
            \textbf{Технології:} Spring, Spring Boot, Spring Security, React, Firebase, Next.js, Redux, JUnit, Mysql, PostgreSQL, Git, Github, Keycloak, Vim, Docker, Docker Compose, Linux
        \end{onecolentry}

    \section{Додатково}

    \begin{onecolentry}
Це резюме створене в \LaTeX{}! Я захоплююся вивченням нових технічних інструментів, які роблять програмування цікавішим та ефективнішим. Відкривши для себе Vim, я поринув у цілий новий світ — навіть після двох років використання продовжую відкривати нові комбінації та вдосконалювати свої навички. Постійний розвиток приносить мені величезне задоволення, надихає на навчання та мотивує професійно зростати.
    \end{onecolentry}

    \vspace{0.1 cm}
    \begin{onecolentry}
        Код цього резюме можна знайти на моїй сторінці github \texttt{:)}
    \end{onecolentry}

    \vspace{0.1 cm}
    \begin{onecolentry}
    Вільно володію англійською та німецькою
    \end{onecolentry}

    \vspace{0.1 cm}
    \begin{onecolentry}
    Готовий до нових викликів та з радістю обговорю ваші пропозиції
    \end{onecolentry}


\end{document}%

